%----------------------------------------------------------------------------------------
% Packages & config %----------------------------------------------------------------------------------------

\documentclass{article}
\usepackage[utf8]{inputenc}
\usepackage[french]{babel}
\usepackage[T1]{fontenc}
\usepackage{amsmath}
\usepackage{amsfonts}
\usepackage{amssymb}
\usepackage{amsthm}
\usepackage{multicol}
\usepackage[margin=2cm]{geometry}
%----------------------------------------------------------------------------------------
%	Infos du doc
%----------------------------------------------------------------------------------------

\title{Recherche d'information : diversité} 
\date{Janvier 2018} 
\author{
   Loïc Herbelot
   \and
   Sébastien Pereira
  } 

\begin{document}

\maketitle 


\tableofcontents{}

\abstract{
Nous nous sommes intéressés au problème de la diversité des résultats renvoyés par un systèmes de recherche d'information. Nous avons implémenté plusieurs algorithmes célèbres que nous avons testés sur un benchmark \textsc{easyCLEF08}.}

\begin{multicols}{2}

\section{Intro}
[~20 lignes]
\paragraph{Problématique}

[Présenter problématique]

[Problèmes à étudier/résoudre]

\paragraph{Interêts}

\paragraph{Difficultés}

\paragraph{Méthode de résolution}

\paragraph{Plan}
[Annoncer plan]

\section{État de l'art}
20 lignes pour présenter diff méthodes de l'état de l'art
Rester général
10 lignes cluster
10 lignes glouton
[Résumer principaux algos vus en cours, leurs avantages et inconvénients]
\paragraph{Post-retrieval clustering}
Intuitivement chaque cluster représente un sous-thème d'une requête.

k-means ?

Clustering hiérarchique ?


\paragraph{Un algorithme glouton}
Algorithme simple à comprendre


\section{Proposition}
[Décrire algos qu'on veut programmer, expliquer choix]
Hypothèse sur un ou deux algos de clustering, montrer que c'est meilleur que l'état de l'art/ la baseline

\section{Expérimentations}
\paragraph{Présentation des données easyCLEF08}

\paragraph{Présentation des mesures d'évaluation}

\paragraph{Résultats}
Parler de la baseline
[Ne pas oublier d'étudier les paramètres]

\section{Conclusion}
[Perspectives et ouvertures]

Biblio

\end{multicols}
\end{document}